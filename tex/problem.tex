\subsection{Problem}\label{sec:problem}

% Uncommon scenarios (good: amazing strain, bad: disease) are hard to
% 1) reproduce, due to multiple factors (such as complexity of the environment or farmer’s decisions)
% 2) prevent, due to lack of data and enough experiences 

% A possible solution is data collection and analysis.
% Currently farmers and researches:
% - collect data from non-connected expensive hardware
% - manually organize data on notebooks and spreadsheets
% - build ad-hoc models for analysis and hypothesis that are biased by previous experiences and the available hardware (correlations are manually inferred)

Two of the most important open problems in dealing
with living organisms are \emph{prevention} and \emph{reproducibility}.

Prevention means being able to stop something from happening. Examples include
the outbreak and spread of diseases (e.g., peronospora~\cite{palti1980downy}) or the exhaustion of resources (e.g., water, fertilizer, etc.).
Being able to forecast such harmful scenarios in advance gives the users the possibility to be prepared and avoid financial losses. Accurate forecasting methods~\cite{box2005statistics} have been proven to be extremely useful for domains
involving living organisms, e.g.,
plant epidemiology~\cite{campbell1990introduction,huber1992modeling},
energy consumption~\cite{neto2008comparison}
or productivity~\cite{nahmias2009production, montgomery1990forecasting}.

Reproducibility is the ability to get the same results based on data previously gathered.
Reproducibility is central in the scientific method~\cite{repko2008interdisciplinary} and many communities, from farmers~\cite{discoverableagriculture} to computer science researches~\cite{peng2011reproducible}, are
becoming sensitive to the ability of being able 
to precisely replicate experiences.
The systematic reproduction of experiments also allows users
to investigate the causes behind rare events and possibly replicate them.
Examples are the exceptional maturation of a vegetable or the outbreak of diseases. 

At the core of prevention and reproducibility are \emph{data collection} and \emph{analysis}~\cite{hey2009fourth,chen2014big}.
Data collection is the process of gathering and measuring information on targeted variables in an established systematic way. Data acquisition enables the analysis of an outcome and the answering of relevant questions,
hence the prevention and reproduction of particular scenarios.

Growers and researchers often manually
collect and organize data in notebooks and spreadsheets.
The same data are then used to build ad hoc models for analysis and hypothesis that are biased by previous experiences and available hardware. Moreover, data is frequently collected from non-connected expensive hardware.

% Nick: Is there any prior work on this front? there has to be something. you should mention any related projects, cite them, and say why they don't solve the whole problem, relating to the points you mention above.

The lack of a rigorous, structured, and systematic
data collection and analysis might penalize users and
prevent them from optimizing their growing environments.


