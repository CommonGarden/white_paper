% Solution
% Smart environment technology:
% 1) Data collection: Modular sensors (i.e., any sensor can be plugged into our framework)
% 2) Data analysis: Data are automatically 1) collected 2) organized 3) analyzed.
% 3) Actuation: Automated decision trigger actuators that control the environmental variables 
% 4) Remote supervision

% Feedback loop graph

\subsection{Solution}\label{sec:solution}

\emph{Smart environments} are physical worlds
that are richly and invisibly interwoven with sensors, computational elements, and actuators~\cite{weiser1999origins}.
A smart environment usually encompasses three main elements:

\begin{enumerate}
    \item {\bf Data Collection} - Sensors are devices whose purpose is to detect changes
    in their environment and send information to other
    modules (e.g., computers). The data collected by sensors, that can be seen as the lifeblood of a smart environment;
    % Common Garden builds \emph{modular} sensors, i.e.,
    % minimal devices that can be easily plugged and integrated into any environment. 
    
    \item {\bf Data Analysis} - Computational elements are the brain of smart environments. Their task is to automatically analyze data generated by sensors and take decisions.
    The analysis and decisions are done
    by algorithms coming from Machine Learning (ML)~\cite{nasrabadi2007pattern, andrieu2003introduction} --- often erroneously called Artificial Intelligence (AI)~\cite{AIvsML} --- and Control Theory~\cite{lee1967foundations, boyd1994linear}, respectively.
    
    \item {\bf Actuation} - The control decisions taken by the computational brain are sent to \emph{actuators}, that are components responsible for moving or controlling a system that might alter the state of the smart environment (e.g., opening or closing a valve).
    % Similarly to sensors, Common Garden build modular and easy to integrate actuators specifically designed to interact into a smart environment.
\end{enumerate}

The client (grower, researcher) can remotely interact with the smart environment via an intuitive
application that gives access to all data and devices.
This enables remote and constant supervision regardless 
the client's location.

Grow-IoT comes with a suite of tools for monitoring, in real time
all the data coming from the environment: that is sensed data,
decisions statuses, and states of activators.\cite{CommonGarden}


% Nick: now seems like the opportunity to describe the system, before moving on to discuss its benefits. it is difficult (impossible) to understand the benefits without the context of how the system works. 

