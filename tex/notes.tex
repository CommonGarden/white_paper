Problem

Uncommon scenarios (good: amazing strain, bad: disease) are hard to
1) reproduce, due to multiple factors (such as complexity of the environment or farmer’s decisions)
2) prevent, due to lack of data and enough experiences 

A possible solution is data collection and analysis.
Currently farmers and researches:
- collect data from non-connected expensive hardware
- manually organize data on notebooks and spreadsheets
- build ad-hoc models for analysis and hypothesis that are biased by previous experiences and the available hardware (correlations are manually inferred)

Solution
Smart environment technology:
1) Data collection: Modular sensors (i.e., any sensor can be plugged into our framework)
2) Data analysis: Data are automatically 1) collected 2) organized 3) analyzed.
3) Actuation: Automated decision trigger actuators that control the environmental variables 
4) Remote supervision

Feedback loop graph


Benefits
1) Farmers can buy the sensors and actuators that they need rather than expensive all-inclusive hardware. This implies:
	- Cost reduction (buy only what you need)
	- Extensibility (buy whenever you need/can and integrate with new designs)
	- Longevity of equipment (software/hardware is upgradable anytime and not dependent on company choices)
2.1/2) Structured data organization
	- Farmer doesn’t need to think about data collection nor to hire a data analyst (costs reduction)
	- Data are stored and collected consistently and interoperable fashion
		- easy to visualize and represent
		- easy to analyze/parse by ML systems
		- transfer and share - benefits the whole community
		- market place
2.3) Analysis is automatically carried out by machine learning systems 
	- Optimization (minimize resources consumption and maximize phenotypic outcomes)
	- Forecast (capture critical scenarios early -> minimize losses)
	- Unbiased data correlation inference
3) Actuators are automatically triggered by farmer configurations/inferred models 
	- Unsupervised management (less employees, larger scale) (24h/day, 365 days/year)
	- Quality consistency. Reproducibility.
	- Human error reduction 
	
4) The user can monitor and control the growth system remotely
    - All data and decisions are available anytime everywhere 
    - Focus on simple/easy to use UIs for non-technical users


Architecture/Implementation

1) Hardware
2) Software


Market research/Use cases

Conclusion and Future Directions