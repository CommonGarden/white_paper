% Benefits
% 1) Farmers can buy the sensors and actuators that they need rather than expensive all-inclusive hardware. This implies:
%     - Cost reduction (buy only what you need)
%     - Extensibility (buy whenever you need/can and integrate with new designs)
%     - Longevity of equipment (software/hardware is upgradable anytime and not dependent on company choices)
% 2.1/2) Structured data organization
%     - Farmer doesn’t need to think about data collection nor to hire a data analyst (costs reduction)
%     - Data are stored and collected consistently and interoperable fashion
%         - easy to visualize and represent
%         - easy to analyze/parse by ML systems
%         - transfer and share - benefits the whole community
%         - market place
% 2.3) Analysis is automatically carried out by machine learning systems 
%     - Optimization (minimize resources consumption and maximize phenotypic outcomes)
%     - Forecast (capture critical scenarios early -> minimize losses)
%     - Unbiased data correlation inference
% 3) Actuators are automatically triggered by farmer configurations/inferred models 
%     - Unsupervised management (less employees, larger scale) (24h/day, 365 days/year)
%     - Quality consistency. Reproducibility.
%     - Human error reduction
% 4) The user can monitor and control the growth system remotely
%   - All data and decisions are available anytime everywhere
%   - Focus on simple/easy to use UIs for non-technical users

% Nick: it is unclear to me if this section is about the benefits of the system, or a description of how the system works. right now it is a bit of both. i would pull these two things apart a bit, for the sake of narrative.

\subsection{Benefits}\label{sec:benefits}

Smart environments can benefit growers in several ways.

\paragraph{Hardware}
\emph{Minimal} and \emph{modular} 
sensors and actuators rather than complex all-inclusive devices.
This benefits the farmer is several ways:
\begin{enumerate}
    \item \emph{Cost reduction} - The grower buys only
    what is really needed.
    \item \emph{Extensibility} -  The grower can buy any device whenever is needed and dynamically adapt the smart environment complexity to the scale of the farm.
    \item \emph{Longevity} - The equipment is upgradable at anytime.
\end{enumerate}

\paragraph{Software}
Much of the data is automatically collected without human intervention. This relieves the grower from the tedious job of manually collecting numbers and enables
the systematic, consistent, and interoperable organization of the gathered information. 

% There are still means of providing human input such as making notes in the event log or uploading pictures of the target organism through their mobile device or a computer.

Properly organized data are easier to visualize and represent, easier to be analyzed and parsed by machine learning systems, and easier to transfer and share
across different farmers. Moreover, a standardized
data collection can lead to a potential market place
where farmers can share knowledge about their crops.
This benefits both the single and the community.

\paragraph{Management}
The most important advantage provided by Common Garden
is the way how smart environments are managed.
The whole processed of data collection, analysis, 
decision making and actuation are automatized and data-driven.


The analysis of data is automatically carried out by \emph{machine learning systems}, i.e.,
a suite of algorithms that can learn and make predictions on data without being explicitly programmed.
The are several benefits in using machine learning instead of manual analysis. 
First, time and money consumption. The workload is 
delegated to the machine instead of being scrutinized by the farmer or by a hired data analyst.
Second, mathematical optimization.
Machine learning algorithms are the result of years of
research and fine tuning. These algorithms are designed to find the optimal solutions to problems that can be solved by looking at data. Some optimization examples are the reduction of resources consumption, such 
as water or fertilizers, or the maximization of phenotypic outcomes. A machine learning algorithm can
find the optimal solution where a human might not see it.
Third, there is a subtle advantage in using machine learning: machines are unbiased.
They exclusively rely on data. A farmer might be influenced by previous experience or other factors.
Objective data-driven decisions taken by machine learning systems can be more precise and
effective than human subjective choices.

The automatized supervision provided by machine learning implies less human intervention
which benefits the farmer who can hire less employees, scale better its farm, and rely on constant
supervision 24h/day 365days/year.

Despite the advantages provided by automation, the farmer should not
lose control on its crops. For this reason, Common Garden develops a monitoring software 
through which the farmer can remotely inspect the status of the farm. The monitoring application
can be used to check the data, the decisions that the machine learning systems are about to take, or
schedule and run new processes. The user interface of this application is purposely designed to 
be simple and easy-to-use for non-technical users. 

%     - Optimization (minimize resources consumption and maximize phenotypic outcomes)
%     - Forecast (capture critical scenarios early -> minimize losses)
%     - Unbiased data correlation inference

% 3) Actuators are automatically triggered by farmer configurations/inferred models 
%     - Unsupervised management (less employees, larger scale) (24h/day, 365 days/year)
%     - Quality consistency. Reproducibility.
%     - Human error reduction 
    
% 4) The user can monitor and control the growth system remotely
%   - All data and decisions are available anytime everywhere
%   - Focus on simple/easy to use UIs for non-technical users